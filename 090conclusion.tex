We have shown how mode-destination accessibility can be used to link post-earthquake infrastructure damage to the impact on human welfare and enables identifying at-risk geographic and demographic groups in a region. 
%Here we have shown a framework for coupling mode-destination accessibility with a quantitative seismic-risk assessment to identify at-risk populations and measure the accompanying impacts on human welfare.
Adopting this state-of-the-art performance metric from the urban planning community, we have illustrated its use for seismic risk assessment and mitigation through a case study of the San Francisco Bay Area. For the case study, we considered a set of 40 hazard-consistent earthquake scenarios, ground-motion intensity maps, damage maps, and corresponding annual rates of occurrence. For each damage map, we performed a detailed activity-based travel model calculation that includes the road network, transit networks, walking and biking options, variable travel demand, and mode choice. We used this data and model to compute the mode-destination accessibility, a performance measure for each community and each socio-economic group (defined by income class and car ownership). 


%, we analyzed impact with a high-fidelity, activity-based travel model that includes the road network, transit networks, walking and biking options, variable travel demand, and mode choice. 
%
%Furthermore, we have proposed a model that captures transport mode choice and the interdependencies of the roads and transit systems. We have nested this network performance model within an event-based probabilistic seismic risk framework. 
%%example
%Using a set of 40 hazard-consistent earthquake scenarios, ground-motion intensity maps, and damage maps, we analyzed impact with a high-fidelity, activity-based travel model that includes the road network, transit networks, walking and biking options, variable travel demand, and mode choice. 
%%In the case study, we first simulated a large set of earthquake scenarios, ground-motion intensity maps, and damage maps. Then, we used optimization to select a subset of the maps. After that, for each of the selected maps, we processed the data for analysis in a high-fidelity, activity-based travel model that includes the road network, transit networks, walking and biking options, variable travel demand, and mode choice. 
%%First, we simulated earthquake scenarios. Adding on spatial correlation, we then simulated ground-motion intensity maps. From these, we generated structural-component and network-component damage maps. We then computed basic network performance (travel time delay) with an efficient travel model. Using an optimization procedure, we selected a subset of these maps for modeling in a high-fidelity transportation model used by the local transportation authorities. The key differences, however, are a) we simulated earthquake damage to bridges, roads, and transit lines, and b) we automated the procedure to run many events in an event-based probabilistic risk framework. 
%We used this data and model to compute the mode-destination accessibility, a state-of-the-art performance measure for each community and each socio-economic group (defined by income class and car ownership). 

%WIthin this broader context, we focus on three communities whose experiences after future earthquakes are expected to differ considerably.

%although we saw this,
We saw stark differences in accessibility from location to location. The far East Bay, south San Jose and some communities south of San Francisco were particularly at risk. We found that these geographic trends persisted across income classes and car ownership groups. Nonetheless higher income households with more cars than workers had higher average accessibility losses than other socio-economic groups. One key reason is the geographic clustering of these households in higher-risk areas. Another factor is that these households tend to take longer daily trips, thus crossing more roads and bridges and possibly increasing the likelihood of disruption.

%despite these differences, ...
This study also demonstrated that travel modes shift after an earthquake, and communities that can more easily adjust are predicted to experience lower post-earthquake losses in accessibility. The results suggest that the walkability of a community, as measured by the percentage of pre-earthquake trips by foot, is closely linked to reduced accessibility risk. 
%We also found that one adaptation measure after major earthquakes is an increased likelihood to walk or bike. 
We also found that in almost all of the simulated earthquake events, the transit system is predicted by this model to be severely impacted. The result is a reduced mode share for transit and increased trips by the other modes (car, walking, and bike). Thus, this study suggests that not modeling transit disruption can lead to a nonconservative estimate of seismic risk of transportation systems. The model shows, however, that when transit is not damaged---which is very rare for this case study---ridership increases.

%in conclusion, xxx has yy important applications:
In conclusion,  mode-destination accessibility offers important insights into the relationship between damage to physical infrastructure and impacts on human welfare. 
Using a detailed transportation network model, computationally efficient analysis strategies, and this refined measures of impact, we obtain new insights about users' risk, and obtain metrics that are usable by urban planners responsible for long-term management of  transportation systems.
This approach provides a foundation for future work in risk mitigation and policy to reduce the vulnerability of at-risk communities. It  suggests that initiatives making communities more conducive for cycling and walking to work can increase resiliency to disasters. It also provides a method to quantify societal benefits of upgrading various aspects of a regions' transportation systems.

%increasing bike friendliness and walkability can increase resiliency.