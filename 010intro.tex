Seismic risk assessment in earthquake engineering tends to focus on buildings, bridges, and the performance of infrastructure systems. For example, for measuring the performance of transportation systems, researchers typically use engineering-based metrics such as the post-earthquake connectivity loss, which quantifies the decrease in the number of origins or generators connected to a destination node~\cite[e.g.,][]{duenas-osorio_seismic_2007,lee_post-hazard_2011}, or the post-earthquake travel distance between two locations of interest~\cite[e.g.,][]{chang_probabilistic_2000}. These frameworks have provided insight into seismic vulnerability and possible risk mitigation. However, the link to the human ramifications can be limited. 

In the field of vulnerability sciences, researchers have long stressed the importance of the impact on human welfare from earthquakes. For example, Bolin and Stanford write that,```Natural' disasters have more to do with the social, political, and economic aspects than they do with the environmental hazards that trigger them. Disasters occur at the interface of vulnerable people and hazardous environments"~\cite{bolin_northridge_1998}. A recent World Bank and United Nations report echoed this idea that the effects on human welfare turn natural hazards into disasters~\cite{the_world_bank_and_the_united_nations_natural_2010}.

%
%, a sentiment echoed by the  
%and furthermore that the study of earthquakes "is not an isolated specialty, but [is] necessarily connected to...issues of development, environmental sustainability, urban geography, political ecology, and critical social theory." 
Historical events emphasize the complex social effects of earthquakes. For example, on one hand the 1994 Northridge earthquake caused major damage to nine bridges, which, while significant, represented only 0.5\% of the bridges estimated by Caltrans to have experienced significant shaking~\cite{california._dept._of_transportation._post_earthquake_investigation_team_northridge_1994}. On the other hand, over half of businesses reported closing after the earthquake with 56\% citing the ``inability of employees to get to work" as a reason~\cite{tierney_business_1997}. Furthermore, the total economic cost of transport-related interruptions (``commuting, inhibited customer access, and shipping and supply disruptions") from this earthquake is estimated at 2.16 billion USD (2014)~\cite{gordon_transport-related_1998}, using the consumer price index to account for inflation~\cite{united_states_department_of_labor_guide_2014}.
%  %$1.5 billion, or 27.3%
%of all local business interruption $1.5 billion, or 27.3%
%of all local business interruption
%Additionally, the complex social causes and impacts of this earthquake have been well-studied in the sociology field~\cite{bolin_northridge_1998}.


%~\cite{tierney_foreshadowing_2006}, the author writes that "Disasters result not from physical disaster agents, such as hurricanes, tornados, and earthquakes, but rather from the juxtaposition of three factors: (1) the disaster agent itself--whether a hurricane, earthquake, tornado, or some technological or human-induced event; (2) the physical setting affected by the disaster, including characteristics of the built environment (e.g., structures not built to survive the physical impact of the disaster agent) and environmental features that serve to either mitigate the effects of disasters or make them more severe (e.g., diminished wetlands that could have cushioned the impacts of Katrina); and (3) population vulnerability, a complex construct that includes such factors as; proximity to physical disaster impacts; material resources (income and wealth); race, ethnicity, gender, age; knowledge concerning recommended safety measures; and factors associated with social and cultural capital" and 


An emergent trend in earthquake engineering related to the social impacts is measuring the cumulative extra time needed for travel after an earthquake, sometimes called travel time delay~\cite[e.g.,][]{kiremidjian_seismic_2007,jayaram_efficient_2010}. This performance measure captures basic re-routing due to road closures and enables identifying roads more likely to be very congested.  Travel time approximately measures one aspect of impact on people, but does not capture the fact that some destinations and trips have higher value than others. Furthermore, this approach measures the impacts by focusing on aggregate regional effects rather than individual communities and demographic groups. Some recent work has looked at other metrics, such as the qualitative criteria-based metric  ``disruption index''~\cite{oliveira_concept_2012}. However, this does not provide a quantitative link between the technical impact and the human ramifications. Other work has looked at resiliency, but defined it in pure engineering terms, such as percentage of a simplified road network that is functional~\cite{bocchini_restoration_2012}. Outside of transportation systems, some researchers have investigated the interplay between earthquake damage, such as damage to water networks, and the usability of houses and other buildings; this represents an important first step~\cite{cavalieri_quantitative_2012,khazai_social_2011,burton_social_2014}.

%GEM
%For a complete picture it is essential to understand also the socio-economic characteristics of populations exposed to earthquake threats, and to meaningfully combine that information with estimates of seismic hazard, exposure, probabilities of loss of life, and damage to property, achieving an integrated and holistic estimate of risk in an area.
%social vulnerability 
%The Social Vulnerability Index (SoVI) (Cutter et al. 2003),
%The Disaster Risk Index of the UNDP;
%The Urban Seismic Risk Index by Carre�o et al., 2007; 2012
%

%I am sitting in the emergency room with my mom and don't have all the info in front of me. Try searching under Cathleen Tierny (or Thierny). She has done a lot of work on the socio economic consequences from earthquakes. She is the director of the earthquake center at u of Colorado. Another names that pops in my head is Bill May - look at PEER reports.
% this work measures the impacts by focusing on aggregate regional effects, rather than on individual communities and demographic groups.
%
%previous research on the long-term economic consequences of 
%disasters (see, for example, Friesema, et al., 1979 and Rossi et 
%al., 1983) measured post-disaster economic well-being by focusing 
%on aggregate community effects, rather than on victimized 
%businesses.

%Furthermore, this metric generally assumes a given pattern of travel demand between cities, usually pre-earthquake conditions~\cite[e.g.,][]{jayaram_efficient_2010}, but refined by~\cite{kiremidjian_seismic_2007}, which inputted a pre-determined travel demand pattern based on a historical event.  Thus, travel time increase as discussed in the literature typically does not capture the possibility that travel demand may change depending on the magnitude of earthquake-related damages. 
%oliveira_concept_2012 : disruption index to look at consequences arose a few criteria, such as food availability. Qualitative score measuring disruption.

%bolin_northridge_1998: 'Natural' disasters have more to do with the social, political, and economic aspects than they do with the environmental hazards that trigger them. Disasters occur at the interface of vulnerable people and hazardous environments. This book concentrates on the social aspects of disaster, focusing on the most expensive disaster to date in US history, the Northridge earthquake of 1994, to examine the facets of vulnerability and post-disaster recovery strategies. Surveying the historical and contemporary aspects of life in Southern California the author explains how vulnerability to disaster has been shaped by more than a century of immigration, urbanization, environmental transformations, and economic development." He writes, that the study of disasters "is not an isolated specialty, but [is] necessarily connected to...issues of development, environmental sustainaiblity, urban geography, political ecology, and critical social theory."

%bruneau_framework_2003: just gives a high-level view: These four dimensions of community resilience�technical, organization, social, and economic (TOSE). social and economic performance measures can be defined that refer to the ability of the community to withstand and recover quickly from the di- saster. social such as avoidance of casualties or plans and resources to meet community needs. technical such as al technology needed for command, control, coordination and critical response tasks is operational. 

%tierney_foreshadowing_2006 (katrina sociology): "Disasters result not from physical disaster agents, such as hurricanes, tornados, and earthquakes, but rather from the juxtaposition of three factors: (1) the disaster agent itself--whether a hurricane, earthquake, tornado, or some technological or human-induced event; (2) the physical setting affected by the disaster, including characteristics of the built environment (e.g., structures not built to survive the physical impact of the disaster agent) and environmental features that serve to either mitigate the effects of disasters or make them more severe (e.g., diminished wetlands that could have cushioned the impacts of Katrina); and (3) population vulnerability, a complex construct that includes such factors as; proximity to physical disaster impacts; material resources (income and wealth); race, ethnicity, gender, age; knowledge concerning recommended safety measures; and factors associated with social and cultural capital, such as routine involvement in social networks that can serve as conduits for information and mutual aid, as well as knowledge that..." REALLY GOOD REVIEW OF BOLIN's BOOK!!!!!!

%tierney_impacts_1995: in north ridge, over half of businesses closed for some period of time, with 56% giving the "inability of employees to get to work" as a reason.


%Previous research on the long-term economic consequences of 
%disasters (see, for example, Friesema, et al., 1979 and Rossi et 
%al., 1983) measured post-disaster economic well-being by focusing 
%on aggregate community effects, rather than on victimized 
%businesses. There has been virtually no systematic social science 
%research that looks at how individual businesses cope with the 
%recovery process and what, if any, long-term consequences result 
%from disaster victimization


In contrast to the work on transportation-related seismic risk, urban planning has a long tradition of studying the impact on people of events and policy~\cite{unwin_town_1909,chapin_urban_1970}. Accessibility is one metric popular in urban planning to measure the impact of different transportation network scenarios, and it measures how easily people can get to desirable destinations~\cite{niemeier_accessibility:_1997}. This ability to travel easily is considered a measure of social impact~\cite[e.g.,][]{kennedy_comparison_2002}. Furthermore, accessibility, by definition, quantifies one key aspect of human welfare~\cite[e.g.,][]{niemeier_accessibility:_1997}. Within urban planning, accessibility has been measured in many ways, including individual accessiblity, economic benefits of accessiblity, and mode-destination accessibility~\cite{geurs_accessibility_2004}. The mode-destination accessibility is computed by taking the  log value of the sum of a function of the utilities of each destination over all possible destinations and travel modes, where the utility decreases if getting to that destination is more costly or time-intensive~\cite{handy_measuring_1997}. This choice of accessibility definition is particularly applicable to quantifying the impacts of catastrophes, such as earthquakes, because certain destinations might be more critical for people in certain locations or from different socio-economic groups (such as low income or high income). 

%In addition to the challenge of linking engineering damage to people, another less-studied topic in the literature is how to model reasonably comprehensive transportation networks for risk assessment. 
While recent work has investigated the interdependencies between different infrastructure networks, such as electric power and water distribution~\cite{hernandez-fajardo_probabilistic_2013,hernandez-fajardo_sequential_2011,duenas-osorio_seismic_2007},  a less well-understood topic is the interdependencies within the transportation system itself. For example, the collapse of a highway bridge may close a transit line if the bridge crosses the transit line. Furthermore, the majority of work to date assumes that travel demand and mode choice will remain unchanged after a future earthquake, which historical data suggests is not the case~\cite{gordon_transport-related_1998}. A first step towards considering variable demand is work in the literature that varies demand by applying a constant multiplicative factor on all pre-earthquake travel demand~\cite{kiremidjian_seismic_2007}. Thus, the prior work suggests three areas of further investigation: 1) the risk of post-earthquake accessibility losses for different people and communities in a region, 2) the impact to the risk assessment results of modeling interdependent transit systems, and 3) the consequences of capturing varying travel demand and different travel modes in the analysis. 

In this paper, we develop a framework for coupling mode-destination accessibility with a quantitative seismic-risk assessment to identify at-risk populations and measure the accompanying impacts on human welfare. 
%In this chapter, we show how mode-destination accessibility quantitatively links transportation network damage to the impacts to human welfare in the weeks after an earthquake. 
%This enables identifying particularly at-risk communities.  
We illustrate our approach with a case study of the San Francisco Bay Area transportation network, including highways, local roads, and public transportation lines.
We simulate earthquake scenarios, ground-motion intensity maps, and damage maps. 
%First, we simulate earthquake scenarios. Adding on spatial correlation, we then simulate ground-motion intensity maps. 
%From these, we generate damage maps. 
We then compute basic network performance (travel time delay) with an efficient travel model, which includes highways and major local roads and fixed demand. Using the optimization procedure we proposed in Miller and Baker 2014~\cite{miller_ground-motion_2014}, we select a subset of these maps for modeling in a high-fidelity transportation model used by the local transportation authorities. 
Our high-fidelity model includes damage to bridges, roads, and transit lines, and varies demand using an agent-based model. While these more comprehensive models are already used in practice for general transportation planning, we extend the models to seismic risk assessment by creating an automated method for damaging and analyzing networks, in order to estimate risk in an event-based probabilistic risk framework. 
%The key differences, however, are that we simulate earthquake damage to bridges, roads, and transit lines; and that we automate the procedure to run many events in an event-based probabilistic risk framework. 
%First, we simulate a large set of earthquake scenarios, ground-motion intensity maps, and damage maps. Then, we use optimization to select a subset of the maps. After that, for each of the selected maps, we use a high-fidelity, activity-based travel model to compute the mode-destination accessibility. The model includes the road network, transit networks, walking and biking options, variable travel demand, and mode choice. 
Finally, we analyze the predicted losses in accessibility according to 12 socio-economic groups used by local planners for the case study region, based on  income class,  and ratio of personal vehicles to workers in a household. 
%We find that predicted post-earthquake accessibility losses between people living in geographically-differing communities vary more than the accessibility losses between people of different economic classes living within a given community. One driver of this trend is that people living in communities with higher proportions of foot traffic are predicted to be less impacted by accessibility losses  than people in other communities. The results lay the foundation for targeting risk mitigation using this improved understanding of where the at-risk communities are located and what may drive this risk.

 